%______________________________________________________________________________________________________________________
% @briefLaTeX2e Resume for Kamil K Wojcicki
\documentclass[margin,line]{resume}
\usepackage[colorlinks = true,
linkcolor = blue,
urlcolor  = blue,
citecolor = blue,
anchorcolor = blue]{hyperref}

\usepackage{fontawesome}

% bibliography with mutiple entries
\usepackage{multibib}
\newcites{papers}{\mysidestyle Selected Papers}
\newcites{talks}{\mysidestyle Selected Talks}
\newcites{works}{\mysidestyle Selected Works}

\usepackage{etoolbox}
\BeforeBeginEnvironment{thebibliography}{%
  \let\origsection\section% save original definition of \section
  \let\section\subsection%  make \section behave like \subsection
}
\AfterEndEnvironment{thebibliography}{%
  \let\section\origsection% restore original definition of \section
}

\usepackage{fancyhdr}
\pagestyle{fancy}
\renewcommand{\headrulewidth}{0pt}
\newcommand{\COMMITHASH}{TRAVISCOMMIT}

\fancyhead{}
\fancyfoot{}
\lfoot{\hspace{-\sectionwidth}\footnotesize \href{{https://kratsg.github.io/cv/?utm_source=cv}}{\faicon{link}~Giordon Stark's Curriculum Vitae}}
\rfoot{\footnotesize Built \href{https://github.com/kratsg/cv/actions/runs/\RUNNUMBER}{\today}\ from \href{https://github.com/kratsg/cv/tree/\COMMITHASH}{\COMMITHASH}}

\linespread{1.0}

%______________________________________________________________________________________________________________________
\begin{document}
\name{\Large Giordon Stark}
\begin{resume}

%__________________________________________________________________________________________________________________
% Contact Information
\section{\mysidestyle Contact\\Information}

Giordon Stark                 \hfill \href{mailto:gstark@cern.ch}{\faicon{envelope}~gstark@cern.ch}
\vspace{0mm}\\\vspace{0mm}%
UCSC SCIPP, 1156 High Street  \hfill \href{https://github.com/kratsg}{\faicon{github}~kratsg} {\large\rmfamily\textbullet} \href{https://twitter.com/kratsg}{\faicon{twitter}~kratsg}
\vspace{0mm}\\\vspace{0mm}%
Natural Sciences 2, Room \#337       \hfill \href{https://orcid.org/0000-0001-6616-3433}{\faicon{key}~0000-0001-6616-3433}
\vspace{0mm}\\\vspace{0mm}%
Santa Cruz, CA \ \ \ 95064\hfill\href{https://giordonstark.com/?utm_source=cv}{https://giordonstark.com/}\\
\vspace{-6.5mm}%

%__________________________________________________________________________________________________________________
% Research Interests
\section{\mysidestyle Research\\Interests}
{\small
High Energy Particle Physics, Supersymmetry and Physics Beyond the Standard
  Model, Electroweak Supersymmetry, Boosted Object Reconstruction, Hadronic
  Final States, Jet Substructure, Machine Learning, Embedded Hardware, Trigger.
}
%__________________________________________________________________________________________________________________
% Education
\section{\mysidestyle Education}

\textbf{University of Chicago}, Chicago, Illinois \hfill \textbf{September 2012 -- July 2018}\\
\textsl{Ph.D. Physics}, David Miller\\
%
\textbf{California Institute of Technology}, Pasadena, California \hfill \textbf{ Sep 2008 -- June 2012}\\
\textsl{B.S. Physics}, Kenneth Libbrecht and Harvey Newman
%
% \textbf{Florida Atlantic University} Boca Raton, Florida \hfill \textbf{Aug 2004 -- June 2008}\\
% \textsl{Engineering Scholars Program} -- dual-enrollment during high school\\[5mm]
%
% \textbf{Suncoast Community High School}, Riviera Beach, Florida \hfill \textbf{Aug 2004 -- Dec 2008}\\
% \textsl{Math, Science and Engineering Program}

\section{\mysidestyle Dissertation}

\textbf{\textsl{Ph.D.}} \href{https://kratsg.github.io/thesis/?utm_source=cv}{\faicon{link}}~\textsl{The search for supersymmetry in hadronic final states using boosted object reconstruction}\\
ISBN: \href{https://books.google.com/books?vid=ISBN978-3-030-34548-8}{978-3-030-34548-8}\\[2.5mm]
\textbf{\textsl{B.S.}}\hspace{3mm} \href{https://www.dropbox.com/s/h0mpop96cn563bq/Thesis.pdf?dl=0}{\faicon{link}}~\textsl{Optical Coating Brownian Thermal Noise in Gravitational Wave Detectors}

%__________________________________________________________________________________________________________________
% Professional Experience
\section{\mysidestyle Professional\\Experience}

\textbf{SCIPP}, Santa Cruz, California \hfill \textbf{August 2018 -- July 2025}\\
\textsl{Post-doctoral Researcher (2018-2024), Project Scientist (2024-2025) ATLAS Experiment at CERN}
\begin{list2}
  \item Executive Committee member of the DPF Coordinating Panel for Software and Computing committee (Jan 2025-present): inaugural 2 year term to promote, coordinate, and assist the HEP community on Software and Computing
  \item Analysis Contact (Aug 2024-present): ATLAS analysis focusing on new physics with radiative decays
  \item Analysis Contact (Aug 2019-Dec 2024): ATLAS cross-section measurement of collinear $W$-boson radiation~\cite{ATLAS:2024mow}
  \item Division of Particles and Fields Executive Committee (Jan-April 2024): Formation Task Force for the Coordinating Panel for Software and Computing
  \item Core developer for module QC software common tools: \href{https://pypi.org/project/module-qc-tools}{\texttt{module-qc-tools}}, \href{https://pypi.org/project/module-qc-analysis-tools}{\texttt{module-qc-analysis-tools}}, \href{https://pypi.org/project/module-qc-database-tools}{\texttt{module-qc-database-tools}}, \href{https://pypi.org/project/module-qc-data-tools}{\texttt{module-qc-data-tools}}, and \href{https://gitlab.cern.ch/YARR/localdb-tools/}{\texttt{localDB}}
  \item Editorial Board member (Sept. 2023-April 2024): VBF diHiggs to four $b$-quarks
  \item Referee for JHEP (four papers), SciPost Physics (three papers)
  \item Early Career Scientist Board committee member (Feb 2021-Dec 2022): assist and foster an inclusive environment for the young scientists in the ATLAS collaboration. This work is also coordinated with other ECS groups in ALICE, CMS, and LHCb. The efforts include hosting "ATLAS Induction Day", software tutorials, Career Q\&A with CERN alumni, organizing social events, and raising awareness of issues specifically impacting the young scientists to Management.
  \item Supersymmetry (SUSY) Run-2 Summaries Subconvener (2020-2022): identify, advise, and lead the publication of combinations across SUSY Physics subgroups, and define recommendations, standards, and validation procedures for analyses to use as part of a harmonization effort. Topics include: simplified model combinations of Electroweak and 3rd Generation physics, pMSSM scans, RPC-to-RPV reinterpretations, incorporating Dark Matter constraints into searches for new physics, and Higgs to SUSY.
  \item Common Dark Matter ASG-RECAST Contact (2019-2021): identify, advise, and implement improvements to the analysis preservation effort and liaison between the ATLAS Software Group and the analyses in the Common Dark Matter group
  \item US ATLAS Diversity and Inclusion Committee (2018-2022): advise the US ATLAS organization on how to implement recommendations in the Diversity and Inclusion report, which recommendations to implement, and modify practices and procedures to improve the collaboration environment for everyone
  \item SUSY Combinations Team Contact (2018-2020): providing recommendations for object identification and selection criteria, developing the toolchains necessary to combine analyses, and performing the statistical combinations
  \item SUSY Monte Carlo Production Contact (2018-2020): usher all Monte Carlo simulation production requests through to the production system, provide support for defining generator configurations for physics, and identify/resolve any bugs - both physics and technical.
  \item Core Developer for \texttt{pyhf} (Ref.~\cite{ATL-PHYS-PUB-2019-029}) allowing for preservation and statistical reproduction of likelihoods for searches in ATLAS
  \item Overseeing module testing and organizing cabling assembly for the ATLAS Detector Inner Tracker (ITk) instrumentation upgrades at SCIPP
  \item Coordinated the dedicated software efforts and developed a python library to interface with the ATLAS Detector ITk upgrade database: \href{https://pypi.org/project/itkdb/}{itkdb}
  \item Supporting analyzer for new physics search in SUSY with multiple heavy-flavor and large radius jets \cite{SUSY-2016-10, SUSY-2015-10, ATLAS-CONF-2017-021, ATLAS-CONF-2016-052, ATLAS-CONF-2015-067}
  \item Instructor for various bootcamps and workshops to help develop software expertise in High Energy Physics (HEP) Analysis Preservation~\cite{AwesomeFeb2020,USATLASBootcamp2019}
\end{list2}

%__________________________________________________________________________________________________________________

\section{\mysidestyle Seminars and\\Plenaries}
University of Hawaii \hfill \textsl{January 2025}\\
CEA Paris-Saclay \hfill \textsl{March 2024}\\
University of Arizona \hfill \textsl{February 2024}\\
University of Arizona \hfill \textsl{January 2024}\\
University of Notre Dame \hfill \textsl{January 2024}\\
Karlsruhe Institute of Technology \hfill \textsl{December 2023}\\
University of Victoria \hfill \textsl{November 2023}\\
National Labs and Department of Energy (DOEPy) \hfill \textsl{August 2023}\\
Wayne State University \hfill \textsl{April 2023}\\
Southern Methodist University \hfill \textsl{March 2023}\\
Stony Brook University \hfill \textsl{February 2023}\\
Lund University, Sweden \hfill \textsl{September 2022}\\
University of Washington, Bothell \hfill \textsl{July 2022}\\
University of Pennsylvania \hfill \textsl{February 2022}\\
Lepton-Photon 2022 \hfill \textsl{January 2022}\\
The University of Cambridge, UK \hfill \textsl{January 2022}\\
The University of Tennessee, Knoxville \hfill \textsl{October 2021}\\
ICPS 2021 \hfill \textsl{August 2021}\\
SUSY2019 \hfill \textsl{May 2019}\\
The University of Chicago \hfill \textsl{November 2018}\\
Columbia College Chicago \hfill \textsl{November 2018}\\
BOOST2016 \hfill \textsl{July 2016}

%__________________________________________________________________________________________________________________

% Honours and Awards
\section{\mysidestyle Honours and\\Awards}

LHCP2025 Poster Award \hfill{May 2025}\\
Breakthrough Prize - Fundamental Physics - Laureate \hfill{Apr 2025}\\
UC Santa Cruz Outstanding Postdoctoral Fellow Award \hfill{May 2022}\\
Springer Thesis Award~\cite{Stark2020} \hfill \textsl{August 2019}\\
CERN Fellowship (turned down) \hfill \textsl{April 2018}\\
Nathan Sugarman Award for Excellence in Graduate Student Research \hfill \textsl{May 2017}\\
US ATLAS Outstanding Graduate Student Award \hfill \textsl{June 2016}\\
Young Researchers' Symposium Award for Best Poster Presentation \hfill \textsl{November 2015}\\
Department of Energy, Office of Science Graduate Student Research \hfill \textsl{Oct. 2015 - Jan. 2016}\\
UChicago Excellence in Graduate Teaching nominee \hfill \textsl{April 2015}\\
US LHC Users Association Lightning Round winner \hfill \textsl{November 2014}\\
UChicago Excellence in Graduate Teaching nominee \hfill \textsl{April 2014}\\
UChicago Excellence in Graduate Teaching nominee \hfill \textsl{April 2013}\\
Caltech Excellent TA Award \hfill \textsl{2012}\\
Edward C. and Alice Stone Fellow \hfill \textsl{June 2010}

%__________________________________________________________________________________________________________________

\newpage

\section{\mysidestyle Past\\Professional\\Experience}

\textbf{UChicago HEP}, Chicago, Illinois \hfill \textbf{November 2013 -- July 2018}\\
\textsl{Ph.D. Student, ATLAS Experiment at CERN}
\begin{list2}
  \item Lead analyzer for new physics search in SUSY with multiple heavy-flavor and large radius jets \cite{SUSY-2016-10, SUSY-2015-10, ATLAS-CONF-2017-021, ATLAS-CONF-2016-052, ATLAS-CONF-2015-067}
  \item Worked on electronic instrumentation to improve the ATLAS Trigger system in Run 3 and beyond for boosted objects: \href{https://gfex.cern.ch/}{gFEX: global feature extraction} \cite{Tang:2104248, DPF2017gFEX}
  \item Spearheading the effort for embedded processor design within the ATLAS experiment. This includes developing an OpenEmbedded firmware layer for compiling a linux kernel from scratch to be installed on instrumentation in the ATLAS experiment: \href{https://github.com/kratsg/meta-l1calo}{\faicon{github}~meta-l1calo}
  \item Investigating jet-area based pile-up suppression techniques applied to jets in the forward region of the ATLAS detector in high pileup environments at HL-LHC \cite{HFSF2017}
  \item Performing physics studies for the hardware instrumentation as part of the ATLAS detector upgrade work. These studies include identifying subjets from trigger tower information, pileup mitigation techniques, parameterizing trigger efficiency and rates for the online trigger objects I defined, and prototyping a convolutional neural network using the ATLAS calorimeter data as a 2D image to study trigger-level observables \cite{DPF2017gFEX}
  \item Editor of the gFEX Final Design Report describing the technical requirements and needs of gFEX in the ATLAS Calorimeter ecosystem \cite{Begel:2233958}
  \item Created and maintain an analysis framework for general physics analyses within ATLAS including Standard Model searches, SUSY, Exotics, Higgs, Trigger-Level analyses, Jet Calibration efforts, and more: \href{https://xaodanahelpers.readthedocs.io/en/master/}{xAODAnaHelpers} \cite{giordon_stark_2020_3743307}
  \item Built a python tool to scan the phase-space of an analysis to identify performant variables to discriminate signal over background: \href{https://github.com/kratsg/Optimization}{root-optimize}
  \item Developer of a python framework that combines ROOT and NumPy: \href{http://scikit-hep.org/root_numpy/}{root\_numpy} \cite{Noel_Dawe_2017}
\end{list2}

\textbf{University of Chicago}, Chicago, Illinois \hfill \textbf{June 2012 -- June 2017}\\
\textsl{Graduate Student Teaching Assistant}\\
\textbf{Courses} (teaching materials and reviews available on request)
\begin{list2}
  \item PHY211 -- Advanced Physics Laboratory \hfill \textbf{Fall Term 2016-2017}
  \item PHY225 -- Advanced Electromagnetism \hfill \textbf{Winter Term 2014-2015}
  \item PHY141 -- Advanced Mechanics \hfill \textbf{Fall Term 2013-2014}
  \item PHY131 -- Mechanics \hfill \textbf{Summer Term 2012-2013}
  \item PHY132b -- Special Relativity and Electromagnetism \hfill \textbf{Winter Term 2012-2013}
  \item PHY121 -- Introductory Mechanics \hfill \textbf{Fall Term 2012-2013}
\end{list2}

\textbf{University of Chicago}, Chicago, Illinois \hfill \textbf{June 2014 -- June 2016}\\
\textsl{Bridge Program Tutor}
\begin{list2}
  \item Tutored participants in the program upon request in all currently offered graduate-level Physics courses at University of Chicago
  \item Bridge program helps enhance diversity in the physics graduate education and also provides a bridge to the \textsl{Ph.D} effort
\end{list2}

\textbf{University of Chicago}, Chicago, Illinois  \hfill \textbf{June 2012 -- May 2013}\\
\textsl{Graduate Student Research Assistant} in Ultracold Atomic Physics, \textbf{Supervisor}: Cheng Chin
\begin{list2}
  \item Started a project on trapping of water droplets using temperature gradients at room pressure
\end{list2}

\textbf{Adaptly}, New York City, New York\hfill \textbf{June 2012 -- September 2012}\\
\textsl{Developer}, \textbf{Supervisors}: Sean Shillo, Will Highduchek
\begin{list2}
  \item \url{https://adaptly.com}
  \item Developed projects and implemented infrastructure for the Adaptly Self-Serve platform
  \item Worked with ``Big Data'' for a large portion of my time at Adaptly
\end{list2}

%\textbf{Basic Web Programming}, Caltech \hfill \textbf{Jan 2011 -- June 2012}\\
%\textsl{Student Instructor}, \textbf{Course Sponsor}: Adam Wierman
%\begin{list2}
%  \item \url{http://ugcs.caltech.edu/~kratsg/PA070b} (\textbf{materials available on request})
%  \item This class was taught during Caltech's Winter Term
%\end{list2}

\textbf{Laser Interferometer Gravitational Observatory}, Caltech \hfill \textbf{Sep 2011 -- June 2012}\\
\textsl{Research Assistant}, \textbf{Advisor}: Rana Adhikari
\begin{list2}
  \item Researching the effects of Brownian Thermal Noise and how it relates to the Quality Factors and Loss Angles of thin-film coated mirrors used in LIGO
\end{list2}

\textbf{Laser Interferometer Gravitational Observatory}, MIT \hfill \textbf{June 2011 -- Sep 2011}\\
\textsl{Research Assistant}, \textbf{Advisors}: Sam Waldman, Rai Weiss, Hugo Paris
\begin{list2}
  \item Developed control systems for monitoring the state of the LIGO system via multiple physical chassis setups and software collaborations
  \item Developed software to analyze noise levels in LIGO hardware (capacitative position sensors and various chassis), analyzed noise levels in the hardware to verify its quality before sending it to other LIGO labs in the country
  \item Worked on the feed-forward systems to minimize mechanical vibrations in the system
\end{list2}

\textbf{Computational Physics Lab}, Caltech \hfill \textbf{March 2011 -- June 2011}\\
\textsl{Research Assistant and Computational Specialist}, \textbf{Advisor}: Frank Rice
\begin{list2}
  \item Developed a new version of the Caltech's Sophomore Physics Laboratory Mathematica CurveFit program (program is still being developed; \textbf{code/demonstration available upon request})
\end{list2}

\textbf{Information Systems and Technology}, Caltech \hfill \textbf{March 2011 -- June 2011}\\
\textsl{Teaching Assistant}, \textbf{Course Instructor}: Shuki Bruck
\begin{list2}
  \item Provided 2-hour Office Hour session once a week to assist with homework, answer questions about lectures, and improve students' understanding
  \item Attended lectures, structured and graded homework assignments for $\sim$ 140 students
\end{list2}

\textbf{Submillimeter Wave Observatory}, Caltech \hfill \textbf{June 2010 -- Aug 2010}\\
\textsl{Edward C. and Alice Stone Fellow}, \textbf{Advisors}: Simon Radford and David Miller
\begin{list2}
  \item Designed an optical system that couples the beams from a Fourier Transform Spectrometer to a Bolometer, collimated through a sample, to determine the submillimeter transmittivity of optical materials in broadband wavelengths (500 Gigahertz to 3.5 Terahertz)
  \item Results are employed in the design of more efficient submillimeter instruments around the world
\end{list2}

%\textbf{Basic Web Programming}, Caltech \hfill \textbf{Mar 2010 -- June 2010}
%\textsl{Student Instructor}, \textbf{Course Sponsor}: Mani Chandy
%\begin{list2}
%  \item \url{http://ugcs.caltech.edu/~kratsg/PA070c} (\textbf{materials available upon request})
%  \item Taught students to design web pages using HTML, CSS, JavaScript, PHP in combination with javascript libraries and JSON.
%\end{list2}

%__________________________________________________________________________________________________________________
% Selected Publications
\section{\mysidestyle Publications, Talks, and Works}
\bibliographystylepapers{atlasBibStyleWithTitle}
\bibliographypapers{papers}
\nocitepapers{*}

A full list of publications is available in INSPIRE: \url{https://inspirehep.net/authors/1319078}.

\bibliographystyletalks{atlasBibStyleWithTitle}
\bibliographytalks{talks}
\nocitetalks{ICPS2021, AwesomeFeb2020, USATLASBootcamp2019, DPF2019DnI, SUSY2019, ATLASPnP2018Pileup, HFSF2017, USLUA2017, DPF2017gFEX, BOOST2016}
\bibliographystyleworks{atlasBibStyleWithTitle}
\bibliographyworks{works}
\nociteworks{*}

\section{\mysidestyle Teaching and\\Training}
I am always heavily documentating my work and ensuring maintainability and sustainability. It is crucially important to ensure continuity, both for the future success of the activities I am involved in, but also to delegate and teach others transferrable skills. Below, I just list a quick summary of various classes, workshops, bootcamps, tutorials I have participated or led over my tenure.
\begin{list2}
  \item CAMPFIRE, ``How to do an ATLAS Analysis'' \hfill \textsl{June 2024}
  \item ATLAS ITk Week, ``Tutorial: Access to PDB backup at CERN'' \hfill \textsl{April 2024}
  \item pyhf Users and Developers Workshop, organizer and instructor \hfill \textsl{December 2023}
  \item US-ATLAS Workshop @ SLAC, ``Systematic Uncertainties'' \hfill \textsl{October 2023}
  \item PyHEP, ``pyhf tutorial and exploration'' \hfill \textsl{October 2023}
  \item Reinterpretation Forum, ``Reduce, Reuse, Reinterpret (mapyde)'' \hfill \textsl{August 2023}
  \item CAMPFIRE, ``How to do an ATLAS Analysis'' \hfill \textsl{July 2023}
  \item Reinterpretation Forum, ``MaPyDe + ATLAS SimpleAnalysis'' \hfill \textsl{December 2022}
  \item ATLAS Exotics Workshop, ``Data (Products) Preservation'' \hfill \textsl{September 2022}
  \item DANCE/CoDaS @ Snowmass 2022, instructor \hfill \textsl{July 2022}
  \item ATLAS ITk Week, ``Using the Production Database API'' \hfill \textsl{May 2022}
  \item US-ATLAS Computing Bootcamp, organizer and instructor \hfill \textsl{October 2021}
  \item PyHEP, ``Distributed statistical inference with pyhf'' \hfill \textsl{July 2021}
  \item ATLAS Induction Day and Software Tutorial, ``Intro to pyhf and hands-on'' \hfill \textsl{July 2021}
  \item ATLAS Induction Day and Software Tutorial, ``Using GitLab for Analysis'' \hfill \textsl{July 2021}
  \item SUSY+HDBS+Exotics RECAST Tutorial, organizer and instructor \hfill \textsl{March 2021}
  \item Future Analysis Systems and Facilities, ``Making an Analysis Pipeline'' \hfill \textsl{October 2020}
  \item CMS B2G Workshop, ``pyhf'', \hfill \textsl{September 2020}
  \item US-ATLAS Computing Bootcamp, organizer and instructor \hfill \textsl{August 2020}
  \item ATLAS Canada Computing Workshop, GitLab CI and Statistical Analysis \hfill \textsl{July 2020}
  \item ATLAS Induction Day and Software Tutorial, ``Intro to pyhf and hands-on'' \hfill \textsl{January 2020}
  \item ATLAS Induction Day and Software Tutorial, ``Using GitLab for Analysis'' \hfill \textsl{January 2020}
  \item ATLAS Induction Day and Software Tutorial, ``Intro to pyhf and hands-on'' \hfill \textsl{October 2019}
  \item ATLAS Induction Day and Software Tutorial, ``Using GitLab for Analysis'' \hfill \textsl{October 2019}
  \item USATLAS/FIRST-HEP Computing Bootcamp, organizer and instructor \hfill \textsl{August 2019}
  \item Analysis Systems Topical Workshop, ``Likelihood publishing \& Reinterpretation'' \hfill \textsl{June 2019}
  \item ATLAS Induction Day and Software Tutorial, ``Using GitLab for Analysis'' \hfill \textsl{January 2019}
  \item US-ATLAS Hadronic Final State Forum, ``Jet and MET triggers'' \hfill \textsl{December 2018}
  \item ATLAS Machine Learning Workshop, ``Intro to pyhf and hands-on'' \hfill \textsl{October 2018}
  \item ATLAS Induction Day and Software Tutorial, ``Using GitLab for Analysis'' \hfill \textsl{October 2018}
  \item ATLAS Induction Day and Software Tutorial, ``Using GitLab for Analysis'' \hfill \textsl{July 2018}
  \item ATLAS Exotics Workshop, ``Tips and tricks for GitLab CI'' \hfill \textsl{May 2018}
  \item ATLAS Software Tutorial, ``Using GitLab for Analysis'' \hfill \textsl{April 2018}
  \item ATLAS S\&C Documentation Workshop, ``Modern doc tools and approaches'' \hfill \textsl{December 2017}
  \item UChicago EFI Data Analytics workshop, ``Python, Jupyter, and ROOT'' \hfill \textsl{October 2017}
  \item ATLAS Hadronic Calibration Workshop, ``Analysis Optimization'' \hfill \textsl{August 2017}
  \item ATLAS Software Tutorial, multiple tutorials \hfill \textsl{June 2016}
  \item UChicago/ASC-ANL software tutorial, multiple tutorials \hfill \textsl{March 2016}
\end{list2}

\section{\mysidestyle Outreach}
I am actively involved in many outreach activities for which I donate my time. These activities vary from working at non-profits, to hobbies where I develop free and open-sourced tools, to actual outreach where I describe the work I do in a public setting.
\begin{list2}
  \item Served as Role Model for Deaf Space Camp Unlimited \hfill \textsl{April 2024}
  \item Introduced Particle Physics to Underrepresented Minorities at Conniston Middle School \hfill \textsl{September 2023}
  \item Served as Panel Member for ``Equality, diversity, and inclusion'' at Lepton-Photon Conference \hfill \textsl{January 2022}
  \item Recorded videos for \href{https://microcosm.web.cern.ch/en}{CERN Microcosm exhibit} in American Sign Language. One of the videos is on \href{https://www.youtube.com/watch?v=BaGjAruqFec}{YouTube}
  \item Expanded the language access of sign language users for Physics via \url{https://aslcore.org} \hfill \textsl{July 2019}
  \item Lobbied Senators and Congressmen to support strong funding for U.S. Particle Physics programs, based on the P5 report, in Washington D.C. \url{https://www.usparticlephysics.org/strategy.html}
  \item Working on SignsFive, an online dictionary for Science, Tech, Engineering, and Math sign language videos to be stored, uploaded, and searched through: \url{http://survey.signsfive.com}
  \item Developed an application that allows small-budget and non-profit theaters to provide free captioning services for their patrons: \url{https://github.com/kratsg/captionator}
  \item Organizing and advocating for accessible theater in Chicago: \url{http://www.chicagoplays.com/access.html} (2013-present)
  \item Volunteered my time to other activists in Chicago who need technical expertise: \url{https://chihacknight.org/} (2015-2018)
\end{list2}

\section{\mysidestyle Mentorship}
I enjoy mentoring other students and helping them succeed both in specific projects, and just generally. Below are a list of students with some details (year if mentored on a specific project, institutional affiliation, or fellow) that I have mentored:
\begin{list2}
  \item Sam Kelson (undergraduate, 2024), IRIS-HEP Fellow [coffea]
  \item Zach Pizzo (undergraduate, 2024), UCSC [ITk Pixels Upgrade]
  \item Samantha Contreras (undergraduate, 2023-2024), UCSC [ITk Pixels Upgrade]
  \item Marco Frank (undergraduate, 2023-present), UCSC [ITk Pixels Upgrade]
  \item Scott Philips (undergraduate, 2023-present), UCSC [ITk Pixels Upgrade]
  \item Sambridhi Deo (undergraduate, 2023) IRIS-HEP Fellow [REANA]
  \item Keaton Ferguson (undergraduate, technician, 2022-2024), UCSC [ITk Pixels Upgrade, and life]
  \item Sam Roberts (graduate, 2021-present), UCSC [ITk Pixels Upgrade, Higgs, and life]
  \item Bo Zheng (masters, 2020), IRIS-HEP Fellow [Hardware Acceleration with GPUs/TPUs]
  \item Noah Peake (undergraduate, technician, 2019-2022), UCSC [ITk Pixels Upgrade, and life]
  \item Hava Schwartz (graduate, 2019-present), UCSC [Monte Carlo, Higgs, and life]
  \item Nathan Kang (graduate, 2020-present), UCSC [Compressed Electroweak Supersymmetry]
  \item Yuzhan Zhao (graduate, 2019-present), UCSC [Collinear-W Measurement]
  \item Jacob Johnson (graduate, 2019-present) [Compressed Electroweak Supersymmetry]
  \item Dr. Jacob Pasner (graduate, 2018-present), UCSC [Higgs, and life]
  \item Dr. Carolyn Gee (graduate, 2018-2023), UCSC [Higgs]
  \item Dr. Natasha Woods (graduate, 2018-2020), UCSC [Quark/Gluon classification]
  \item Dr. Emily Smith (graduate, 2017-present), UChicago [gFEX, Supersymmetry, and life]
  \item Dr. Michael Hank (graduate, 2017-2019), UChicago [Supersymmetry]
  \item Henry Zheng (undergraduate, 2018), UChicago [gFEX System-on-Chip Development]
  \item Ben Warren (undergraduate, 2018), UChicago [gFEX + Machine Learning]
  \item Brandon Nadal (undergraduate, 2017), UChicago REU MRSEC [gFEX Monitoring]
  \item Natalie Harrison (undergraduate, 2015-2017), UChicago [Supersymmetry, Recursive Jigsaw]
  \item Daniel Sullivan (undergraduate, 2015-2016), UChicago [gFEX Monitoring]
  \item Michael Reid (undergraduate, 2014-2015), UChicago [Boosted Higgs to bb]
  \item Sean Gasiorowski (undergraduate, 2014-2015), UChicago [Boosted objects]
\end{list2}

%__________________________________________________________________________________________________________________
% Languages, Programming, Skills
\section{\mysidestyle Languages}

American Sign Language, English (bilingual), French Sign Language (elementary), British Sign Language (elementary), Italian Sign Language (elementary), Spanish (elementary), French (elementary)

\section{\mysidestyle \faicon{code}~Programming}

\textsl{Full Stack Developer}, C, C++, Perl, Python, \LaTeXe, MySQL, PHP, JavaScript, JSON, HTML, XHTML, XML, CSS, VHDL, Continuous Integration, Version Control, GitHub/GitLab

\section{\mysidestyle Computing Libraries}

Mathematic, Matlab, COMSOL, ROOT, Keras, scikit-learn, TensorFlow, NumPy, SciPy, \texttt{pyhf}, uproot, root\_numpy, rootpy, PyROOT, Matplotlib, pandas, BitBake/OpenEmbedded, Docker, Git, NodeJS, React, jQuery, Bootstrap, pybind11

\section{\mysidestyle Skills}

Effective communication, public speaking, collaboration, project management, mentoring, adaptability, flexibility, baking

%__________________________________________________________________________________________________________________
% Explaining this CV hosted
% \section{\mysidestyle Continuous Integration}
%
% This curriculum vitae was built automatically using continuous integration that ran from a github repository hosting the latex sources. After building the PDF, this is deployed back to github where it is hosted for free as a static website. The links in the footer on every page provide details about the version of the CV you are reading, when it was built, and where the currently hosted version is. Try the links out!
%
%______________________________________________________________________________________________________________________
\end{resume}
\end{document}
